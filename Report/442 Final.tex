\documentclass[11pt]{amsart}
\usepackage{geometry}                % See geometry.pdf to learn the layout options. There are lots.
\geometry{letterpaper}                   % ... or a4paper or a5paper or ... 
%\geometry{landscape}                % Activate for for rotated page geometry
%\usepackage[parfill]{parskip}    % Activate to begin paragraphs with an empty line rather than an indent
\usepackage{graphicx}
\usepackage{amssymb}
\usepackage{epstopdf}
\DeclareGraphicsRule{.tif}{png}{.png}{`convert #1 `dirname #1`/`basename #1 .tif`.png}

\title{EECS 442 Final Project}
\author{Grant Kennell and Anthony Green}

\begin{document}
\maketitle
\section{Introduction}

Robust facial recognition is one of the standard research problems facing the discipline of computer vision. The difficulty is multifaceted. Important questions include:
1. How can facial recognition techniques deal with variations in pose?
2. How can facial recognition techniques deal with variations in lighting?
3. How can facial recognition techniques efficiently represent faces so as to speed up processing as image sets grow large?
4. How can facial recognition techniques accurately match faces?
The paper whose method we implemented, \textbf{Enhanced Texture Feature Sets for Face Recognition Under Difficult Lighting Conditions}, by Xiaoyang Tan and Bill Triggs, deals with the latter three questions. It s

\section{Methods}
\subsection{Previous Work}
\subsection{Our Work}
\section{Technical Summary}
\subsection{Technical Overview}
\subsection{Implementation Details}
\section{Experimental Results}


\end{document}  